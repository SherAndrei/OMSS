\newpage
\section{Билет 11. Идеальная жидкость. Уравнения Эйлера. Полная система уравнений, описывающая движения идеальной несжимаемой жидкости. Граничные условия.}

\begin{center} \textit{\underline{Идеальная жидкость}} \end{center}

\begin{defn}[]\textbf{Жидкость} или \textbf{газ} -- это среда, в которой \underline{в состоянии покоя} отсутствуют касательные напряжения, т.е. в состоянии покоя вектор напряжений на любой площадке параллелен нормали к площадке: $ \vec{P_n} || \vec{n} $, то есть $ \vec{P_n} = P_{nn} \vec{n} $, где $ P_{nn} $ -- проекция вектора $ \vec{P_n} $ на нормаль $ \vec{n} $ к площадке.
\end{defn}

\textcolor{gray}{В твёрдом теле это не так. Например, тяжёлый твердый кирпич может лежать на наклонной плоскости. На площадках, параллельных плоскости, действуют касательные силы, которые уравновешивают соответствующую составляющую силы тяжести. Жидкий «кирпич» на наклонной плоскости лежать не может, жидкость будет течь.}

\begin{theorem}[Э-124] \textbf{Теорема о давлении} (закон Паскаля, следствие из ЗСКД или формулы Коши (Э-107))
  Если на всех площадках в данной точке вектор напряжений $ \vec{P_n} $ перпендикулярен площадке, то есть $ \vec{P_n} = P_{nn} \vec{n} $, то величина $ P_{nn} $ на всех площадках в данной точке одна и та же.
\end{theorem}

Вводится обозначение $P_{nn} = -p$. Величина $p$ называется \textbf{давлением}.

В любой жидкости и любом газе \underline{в состоянии покоя} вектор напряжений имеет вид $ \vec{P_n} = p \vec{n} $, а компоненты тензора напряжений в декартовой СК \underline{в состоянии покоя}: $ p^{ij} = -p \delta^{ij} $.

В движущихся жидкостях и газах, конечно, могут возникать касательные напряжения.
\textcolor{gray}{Это свойство жидкостей называется вязкостью, а сами касательные напряжения в жидкостях и газах называют вязким трением.}

\begin{defn}[Э-125] Жидкость или газ называются \textbf{идеальными}, если в них не только в состоянии покоя, но и \underline{при движении} отсутствуют касательные напряжения.
\end{defn}

Из определения следует, что компоненты тензора напряжений в декартовой СК для идеальных жидкостей или газа при движении имеют такой же вид, как и в состоянии покоя:
$$\boxed{ p^{ij} = -p \delta^{ij} }$$

\begin{center} \textit{\underline{Уравнения Эйлера.}} \end{center}

Уравнения движения идеальной жидкости или газа называются \textbf{уравнениями Эйлера}.
Их получают из общих уравнений движения с использованием формулы $ p^{ij} = -p \delta^{ij} $.

Уравнения движения любой среды:
$$ \rho \frac{dv^i}{dt} = \rho F^i + \nabla_k( p^{ik} ) $$

Используем, что в идеальной жидкости:
$$
  p^{ik} = -p \delta^{ik} \;\Rightarrow \;
  \nabla_k( p^{ik} ) = \frac{\partial p^{ik} }{\partial x^k} =
  \delta^{ik} \frac{\partial p }{\partial x^k} = \frac{\partial p }{\partial x^i}
$$

Тогда:
$$ \rho \frac{dv^i}{dt} = \rho F^i + \frac{\partial p }{\partial x^i} $$

\begin{theorem}[Э-126]Уравнения Эйлера:
  $$\boxed{ \rho \frac{d \vec{v}}{dt} = \rho F + \Grad p }$$
\end{theorem}

\begin{center} \textit{\underline{Полная система уравнений, описывающая движения идеальной несжимаемой жидкости.}} \end{center}

\begin{defn}[]Жидкость называется \textbf{несжимаемой}, если её плотность в частице при движении сохраняется.
  $$ \rho = \Const \;, \quad \frac{d \rho}{dt} = 0 $$
\end{defn}

Полная система уравнений идеальной жидкости:
$$
  \begin{cases}
    \frac{d \rho}{dt} = \rho \Div \vec{v} \quad        & \text{ --- ур-ние неразрывности } \\
    \rho \frac{d \vec{v}}{dt} = \rho F + \Grad p \quad & \text{ --- ур-ние движения }
  \end{cases}
$$

Полная система уравнений идеальной \underline{несжимаемой} жидкости (Э-127):
$$
  \begin{cases}
    \Div \vec{v} = 0 \quad                             & \text{ --- ур-ние неразрывности } \\
    \rho \frac{d \vec{v}}{dt} = \rho F + \Grad p \quad & \text{ --- ур-ние движения }
  \end{cases}
$$

\begin{center} \textit{\underline{Граничные условия.}} \end{center}

Границы области, занятой жидкостью, бывают двух типов:
\begin{enumerate}
  \item твёрдые, границы тел, погружённых в жидкость (например, стенки трубы, поверхность подводной лодки, движущейся под водой, опоры моста, самолета в воздухе)
  \item свободные, форма которых заранее не известна (например, покрытая волнами поверхность моря)
\end{enumerate}

\begin{defn}[Э-128] Граничное условие на поверхности твёрдых тел в идеальной жидкости называется \textbf{условием непроницаемости}. Это условие означает, что жидкость или газ не проникают внутрь тела и не отрывается от него.
  Для этого необходимо, чтобы скорости жидкости и соответствующей точки тела вдоль нормали к поверхности тела были одинаковы, то есть в точках поверхности тела выполнялось равенство:
  $$ \left. v_n \right|_{\text{на поверхности тела}} = v_{n \; \text{тела}}$$
  где $ v_{n \; \text{тела}} $ -- нормальная составляющая скорости точки поверхности движущегося тела.
\end{defn}

Если тело неподвижно, то условие непроницаемости на его поверхности имеет вид:
$$ \left. v_n \right|_{\text{на поверхности тела}} = 0 $$
