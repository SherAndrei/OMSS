\newpage
\section{Билет 5. Распределение скоростей в малой окрестности точки сплошной среды (Формула Коши-Гельмгольца). Вектор вихря. Потенциальность течения.}

\begin{center}
  \textit{\underline{Формула Коши-Гельмгольца}}
\end{center}

Рассмотрим некоторую точку $M$ сплошной среды с координатами $x^i$ и ее малую окрестность. Пусть точка $M'$ с координатами $x^i + dx^i$ - некоторая произвольная точка из окрестности $M$. Тогда с помощью формулы Тейлора распишем скорость в точке $M'$:
$
  \vec{v} (M') = \vec{v} (x^i + dx^i) = \vec{v}(x^i) + \frac{\partial \vec{v}}{\partial x^i}dx^i = \vec{v} (M) + \nabla_i v_j dx^i \vec{\text{э}}^j = \vec{v} (M) + \frac{1}{2}(\nabla_i v_j  + \nabla_j v_i)dx^i \vec{\text{э}}^j + \frac{1}{2}(\nabla_i v_j  - \nabla_j v_i)dx^i \vec{\text{э}}^j (*)
$, где $\vec{\text{э}}^j$ - контрвариантные вектора базиса. Известно, что $\frac{1}{2}(\nabla_i v_j + \nabla_j v_i) = e_{ij}$ - компоненты тензора скорости деформации.

\defn \textbf{Компоненты тензора вихря} -- $ \omega_{ij} := \frac{1}{2}(\nabla_i v_j - \nabla_j v_i)$

\defn \textbf{Вектор вихря} -- $\omega^k := \frac{1}{\sqrt{g}}\omega_{ij}$, где  $(i, j, k)$ - круговая перестановка из $(1, 2, 3)$.

\state \textbf{Формула Коши-Гельмгольца}: Пусть известна скорость частицы в точке $M$, тогда для частицы в точке $M'$ (из окрестности $M$) верно следующее: $\vec{v}(M') = \vec{v}(M) + e_{ij}dx^i \vec{{\text{э}}}^j + [\omega \times d\vec{r}]$.

Доказательство:
1. Проверим, что:
\begin{equation*}
  \omega_{ij}dx^i \vec{\text{э}}^j = [\omega \times d\vec{r}] = \sqrt{g}
  \begin{vmatrix}
    \vec{\text{э}}^1 & \vec{\text{э}}^2 & \vec{\text{э}}^2 \\
    \omega^1         & \omega^2         & \omega^3         \\
    dx^1             & dx^2             & dx^3
  \end{vmatrix}
  (**)
\end{equation*}

Вычислим компоненту при $\vec{\text{э}}^1$:
$$\omega_{i1}dx^i = \omega_{21}dx^2 + \omega_{31}dx^3 = -\sqrt{g}(\omega^3 dx^2 - \omega^2dx^3)$$

Аналогичным образом получаются остальные компоненты $\vec{\text{э}}^j$, таким образом $(**)$ верно. Подстановкой $(**)$ в $(*)$ получается искомая формула Коши-Гельмгольца.

На основе \textbf{формулы Коши-Гельмгольца} можно сформулировать \textbf{теорему Коши-Гельмгольца}.

\theorem \textbf{Коши-Гельмгольца}: Движение малой частицы сплошной среды можно представить как сумму:
\begin{enumerate}
  \item Поступательного движения со скоростью $\vec{v}(M)$
  \item Вращение как твердого тела с угловой скоростью $\vec{\omega}$
  \item Движения, связанного с деформированием, которое описывается скоростью $e_ijdx^i \vec{\text{э}}^j$
\end{enumerate}

\begin{center}
  \textit{\underline{Вектор вихря}}
\end{center}

Определение \textbf{Вектора вихря} было введено выше, далее рассмотрим некоторые его свойства и утверждения связанные с ним.

\textbf{Механический смысл вектора вихря} следует из формулы Коши-Гельмгольца и заключается в следующем: Если частицы движутся как твердое тело, то $\vec{\omega}$ - это угловая скорость этого тела.

\defn Движение называется \textbf{безвихревым}, если $\vec{\omega} = 0$

\state $\vec{\omega} = \frac{1}{2}rot\, \vec{v}$

Доказательство: По определению ротором вектора $\vec{v}$ называется велчина:
\begin{equation*}
  \vec{v} = \frac{1}{\sqrt{g}}
  \begin{vmatrix}
    \vec{\text{э}}^1 & \vec{\text{э}}^2 & \vec{\text{э}}^2 \\
    \nabla^1         & \nabla^2         & \nabla^3         \\
    v_1              & v_2              & v_3
  \end{vmatrix}
\end{equation*}

Из этого определения видно, что для первой компоненты вектора вихря верно равенство:
$$
  \frac{1}{2}(rot\, \vec{v})^1 = \frac{1}{2 \sqrt{g}} (\nabla_2v_3 - \nabla_3 v_2) = \omega^1
$$

\begin{center}
  \textit{\underline{Потенциальность течения}}
\end{center}


\defn \textbf{Потенциалом скорости} называется такая функция $\varphi$, что для скорости $\vec{v}$ верно:

$$
  \vec{v} = grad\, \varphi\, (\text{то есть}\, v_i = \nabla_i \varphi = \frac{\partial \varphi}{\partial x^i})
$$

\defn Движение называтеся \textbf{потенциальным}, если существует потенциал скорости.
\state $\exists \varphi : \vec{v} = grad\, \varphi$ тогда и только тогда $\vec{\omega} = 0$

Доказательство:
1) $(=>)$ Пусть $\exists \varphi : \vec{v} = grad \, \varphi$, тогда:

$
  \omega_x = \frac{1}{2}(\frac{\partial v_z}{\partial y} - \frac{\partial v_y}{\partial z}) = \frac{1}{2}(\frac{\partial}{\partial y}(\frac{\partial \varphi}{\partial z}) - \frac{\partial}{\partial z}(\frac{\partial \varphi}{\partial y})) = \frac{1}{2} (\frac{\partial^2 \varphi}{\partial y \partial z} - \frac{\partial^2 \varphi}{\partial z \partial y}) = 0
$

2) $(<=)$ Пусть $\vec{\omega} = 0$, тогда рассмотри дифференциальную форму $v_xdx + v_ydy + v_zdz$, она представляет из себя полный дифференциал некоторой функции, тогда и только тогда когда:
$$
  \frac{\partial v_z}{\partial y} = \frac{\partial v_y}{\partial z}, \frac{\partial v_x}{\partial z} = \frac{\partial v_z}{\partial x}, \frac{\partial v_y}{\partial x} = \frac{\partial v_x}{\partial y}
$$

Но эти условия выполнены, если $\vec{\omega} = 0$. Таким образом при $\vec{\omega} = 0$
$v_xdx + v_ydy + v_zdz = d\varphi$, отсюда следует, что

$$
  v_x = \frac{\partial \varphi} {\partial x}, v_y = \frac{\partial \varphi} {\partial y}, v_z = \frac{\partial \varphi} {\partial z}
$$ то есть $\vec{v} = grad\varphi$
