\newpage
\section{Билет 2. Описание деформации. Тензоры Грина и Альманси. Главные оси и главные значения тензоров. Инварианты тензоров.}

\begin{center}
  \textit{\underline{Описание деформации}}
\end{center}

\defn{Деформация} - это изменение длин всевозможных материальных отрезков и углов между ними. При деформациях рассматривают малую окрестность некоторой точки М сплошной среды.

Рассмотрим малую частицу среды - малую окрестность некоторой произвольной точки М. Точки из малой окрестности точки М имеют в начальном состоянии координаты $x_0^i+dx_0^i$, так что если бы мы эту точку приняли за начало координат, то координаты всех точек из ее малой окрестности были бы $dx_0^i$. В конечном состоянии точка М имеет координаты $x^i$, а точки из ее окрестности $x^i+dx^i$ .

$$dx^i=\frac{\partial x^i}{\partial x_0^k}dx_0^k$$
Матрица $\left( \frac {\partial x^i}{\partial x_0^k}\right)$ - матрица дисторсии. Все преобразование чатицы, кроме поступательного переноса вместе с центром называется \textit{дисторсией}

\begin{center}
  \textit{\underline{Тензоры деформаций}}
\end{center}
Чтобы ввести тензоры деформаций, удобно поначалу воспользоваться лагранжевой системой координат $\xi^i$.

Лагранжевы координаты - это параметры, которые для каждой индивидуальной точки фиксированы, и не меняются, что бы с ней ни происходило. Поэтому у точки М координаты $\xi^i$ и в начальном, и в деформированном состояниях - одни и те же, и относительные координаты индивидуальных точек окрестности, то есть $d\xi^i$, -одни и те же. Конечно, это значит, что сама система координат деформируется вместе со средой.

Квадраты длин материальных отрезков, выходящих из точки М, в начальном и конечном состояниях соответственно равны $ds_0^2 = \mathring{g}_{ij}d\xi^id\xi^j$ и $ds^2 = \hat{g}_{ij}d\xi^id\xi^j$, причём величины $d\xi^i$, $d\xi^j$ в обеих формулах одни и те же по определению лагранжевых координат. Рассмотрим разность квадратов длин малых отрезков до и после деформации: $$ds^2 - ds_0^2 = (\hat{g}_{ij} - \mathring{g}_{ij})d\xi^id\xi^j$$
$$\varepsilon_{ij} = \frac{1}{2}(\hat{g}_{ij} - \mathring{g}_{ij})$$

\defn{Тензором деформаций Грина (или лагранжевым тензором деформаций)} называют тензор $$\mathring{\mathcal{E}}=\varepsilon_{ij}*\overrightarrow{\mathring{\text{э}^i}}*\overrightarrow{\mathring{\text{э}^j}}$$ где $\overrightarrow{\mathring{\text{э}^j}}$ контравариантные векторы базиса лагранжевой системы координат в начальном состоянии.

\defn{Тензором деформаций Альманси (или эйлеровым тензором деформаций)} называют тензор $$\hat{\mathcal{E}}=\varepsilon_{ij}*\overrightarrow{\hat{\text{э}^i}}*\overrightarrow{\hat{\text{э}^j}}$$ где $\overrightarrow{\hat{\text{э}^j}}$ контравариантные векторы базиса лагранжевой системы координат в конечном состоянии


\begin{center}
  \textit{\underline{Главные оси и главные значения тензоров.}}
\end{center}

Вводятся единичные векторы базиса $\overrightarrow{e_i}$, направленные по осям лагранжевой системы в деформированном состоянии:
$$\overrightarrow{e_i}=\frac{\overrightarrow{\text{э}_{i}}}{|\overrightarrow{\text{э}_{i}}|}=\overrightarrow{\text{э}_{i}}\sqrt{1+2\mathring{\xi_i}}$$

Система с вектор ами базиса $\overrightarrow{e_i}$ будет главной для тензора Альманси.
Связь между главными компонентами тензоров Альманси и Грина таковы:
$$\hat\xi_i=\frac{\mathring{\xi_i}}{1+2\mathring\xi_i}$$

\begin{center}
  \textit{\underline{Инварианты тензоров}}
\end{center}
Инварианты тензоров используются для получения величины относительного изменения объема $\theta$.
При деформации параллелепипед переходит в параллелепипед с прямоугольными ребрами
$$\theta=\sqrt{1+2\mathring{I_1}+4\mathring{I_2}+8\mathring{I_3}}-1$$
$$\theta=\sqrt{1-2\hat{I_1}+4\hat{I_2}-8\hat{I_3}}-1$$

где $\mathring{I_k},\hat{I_k}$ -первый, второй и третий инварианты тензоров Грина и Альманси:
$$I_1=\xi_1+\xi_2+\xi_3=g_{ij}\xi_{ij}$$
$$I_2=\xi_1\xi_2+\xi_2\xi_3+\xi_3\xi_1=\frac{1}{2}(I_1^2-\xi^{ij}\xi_{ij})$$
$$I_3=\xi_1\xi_2\xi_3=det(\xi_j^i)$$
