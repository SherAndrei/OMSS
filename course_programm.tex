\documentclass[specialist, subf, href, colorlinks=true, 14pt, times, mtpro, final]{report}

\usepackage[utf8x]{inputenc}
\usepackage[english, russian]{babel}
\usepackage[T2A]{fontenc}
\usepackage{amsmath,amsthm,amssymb}
\usepackage {wrapfig}
\usepackage {enumitem}  
\usepackage{graphicx}
\usepackage{multicol}
\usepackage{mathrsfs}
\usepackage{xcolor}
\usepackage{hyperref}
\usepackage{tikz}
\usepackage{pdfpages}


\usetikzlibrary{decorations.pathreplacing}
\usepackage[a4paper, mag=1000, includefoot, left=1.5cm, right=1.5cm, top=1cm, bottom=1cm, headsep=1cm, footskip=1cm]{geometry}
\usepackage{floatrow}
\usepackage{tikz}
\newcommand{\RNumb}[1]{\uppercase\expandafter{\romannumeral #1\relax}}
\usetikzlibrary{graphs}

\theoremstyle{definition}
\newtheorem{defn}{Определение}[section]
\newtheorem{example}{Пример}[section]
\newtheorem{state}{Утверждение}[section]
\newtheorem{theorem}{Теорема}[section]
\newtheorem{lemma}{Лемма}[section]
\newtheorem{axiom}{Аксиома}[section]
\newtheorem{consequence}{Следствие}[section]

\begin{document}
\centering
{\bf Программа курса <<Основы механики сплошных сред>>}
\begin{enumerate}
  \item Понятие сплошной среды. Лагранжево и эйлерово описание движения сплошной среды. Индивидуальная производная.
  \item Описание деформации. Тензоры Грина и Альманси. Главные оси и главные значения тензоров. Инварианты тензоров.
  \item Вектор перемещения. Связь вектора перемещения и тензоров деформации. Случай малых деформаций. Геометрический смысл компонент. Относительное изменение объема. Уравнение совместности малых деформаций.
  \item Тензор скоростей деформации. Его связь с полем скоростей. Кинематический смысл компонент тензора скоростей деформации. Дивиргенция скорости.
  \item Распределение скоростей в малой окрестности точки сплошной среды (Формула Коши-Гельмгольца). Вектор вихря. Потенциальность течения.
  \item Производная по времени от интеграла по подвижному объему. Уравнение неразрывности в переменных Эйлера и Лагранжа.
  \item Изменение количества движения конечного объема сплошной среды. Массовые и поверхностные силы. Вектор напряжения. Тензор напряжения. Механический смысл его компонент. Изменение момента количества движения
  \item Дифференциальные уравнения движения сплошной среды и момента количества движения. Симметрия тензора напряжения. Теорема живых сил.
  \item Изменение энергии в конечном объеме сплошной среды (первое начало термодинамики). Работа и внутренняя энергия. Уравнение притока тепла.
  \item Понятие температуры. Теплопроводность. Цикл Карно. Абсолютная температура. КПД цикла Карно. Второе начало термодинамики. Энтропия.
  \item Идеальная жидкость. Уравнения Эйлера. Полная система уравнений, описывающая движения идеальной несжимаемой жидкости. Граничные условия.
  \item Уравнения движения идеальной жидкости в форме Громеки-Лемба. Интегралы Бернулли и Коши- Лагранжа.
  \item Теоремы о вихрях в идеальной жидкости.
  \item Безвихревое движение идеальной жидкости в трехмерном и в двумерном случаях. Примеры потенциалов. Применение теории функции комплексного переменного для решения задач плоского движения идеальной несжимаемой жидкости. Формула Жуковского.
  \item Совершенный газ. Адиабатический процесс. Полная система уравнений, описывающая движение нетеплопроводного идеального газа. Скорость звука. Число Маха. Критерий сжимаемости в стационарном случае. Квазиодномерные стационарные течения. Сопло Лаваля.
  \item Вязкая жидкость. Закон Навье-Стокса. Уравнения Навье-Стокса. Полная система уравнения, описывающая движение вязкой несжимаемой жидкости. Течение Пуазейля. Течение Куэтта. Закон Фурье. Полная система уравнений, описывающая движение вязкого теплопроводного совершенного газа.
  \item Безразмерные параметры, определяющие характер движения несжимаемой вязкой жидкости. Число Рейнольдса. Предельные случаи. Пограничный слой.
  \item Линейно упругое тело. Модуль Юнга, коэффициент Пуассона, коэффициненты Ламе. Связь этих коэффициентов между собой. Их физический смысл. Уравнения Ламе.
  \item Постановка задач теории упругости в перемещениях и напряжениях. Граничные условия. Принцип Сен-Венана.
  \item Задача об изгибе балки.
  \item Кручение цилиндрических стержней
\end{enumerate}


\end{document}
